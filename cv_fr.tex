\documentclass[11pt,sans,a4paper]{moderncv}

\moderncvstyle{classic}                        % style options are 'casual' (default), 'classic', 'oldstyle' and 'banking'
\moderncvcolor{blue}                          % color options 'blue' (default), 'orange', 'green', 'red', 'purple', 'grey' and 'black'

\usepackage[utf8]{inputenc}
\usepackage[left=0.5cm,top=0.5cm,right=0.5cm,bottom=0,nohead,nofoot]{geometry}
\usepackage[none]{hyphenat}
\setlength{\hintscolumnwidth}{3cm}           % if you want to change the width of the column with the dates

\name{Boris}{SABATIER}
\title{Ingénieur en informatique}
\address{21 rue neyret}{69001 Lyon}
\phone[mobile]{06 50 82 32 97}
\email{boris-cv@sabatier.io}
%\birthdate{Né le 02/08/1988}
\extrainfo{Permis voiture depuis 2006}

\begin{document}
	\makecvtitle
	\section{Expériences}
        \cventry{Depuis Septembre 2016}{Ingénieur R\&D}{Stormshield}{Lyon}{France}{
			\begin{itemize}
                \item Développement d'un outil d'analyse de virus en sandbox
                \item Développement d'un outil simulation des protocoles d'internet
			\end{itemize}
			Technologies : Python, MongoDB, Golang, C, Javascript
        }
		\cventry{Janvier 2015\\à Septembre 2016}{Ingénieur en développement}{Qualistream}{Lyon}{France}{
			\begin{itemize}
                \item Rétro-ingénierie du protocole SCCP/Skinny
                \item Amélioration d'outils de simulation d'équipement téléphonique et informatique
                \item Suivi de l'environnement de production et gestion des incidents client
                \item Montage, calibration et maintenance d'imprimantes 3D
                \item Conception et réalisation mécanique et éléctronique de robot de simulation
			\end{itemize}
			Technologies : C, Yocto/Open embedded, Shell, Nodejs, XMLRPC,  Python, Ruby, SQL
		}
		\cventry{Janvier 2013\\à Janvier 2015}{Ingénieur en développement}{Newbiz conseil}{Lyon}{France}{
			\begin{itemize}
				\item Développement d'applications vocales interactives pour les comptes de Total et GrDF, via Prosodie.
			\end{itemize}
			Technologies : C, Shell, Python, XML, XSL, VXML 
		}
		\cventry{Septembre 2012\\à Janvier 2013}{Ingénieur en informatique Industriel}{ADENEO}{Saint Priest}{France}{
			\begin{itemize}
				\item Développement de logiciels embarqués pour le compte de VOLVO.
			\end{itemize}
			Technologies : C, XML, XSL, CANalyser, outils VECTOR
		}
		\cventry{Septembre 2009\\à Septembre 2012}{Ingénieur en informatique en apprentisage}{Electricité De France}{Villeurbanne}{France}{
			\begin{itemize}
				\item Intégration d'un superviseur de calcul dans un outil de chaînage de calcul scientifique.
				\item Amélioration d'un outil de chaînage de calcul scientifique.
				\item Packaging d'applications pour GNU/Debian.
				\item Développement d'un outil interne de gestion de listes de contacts.
			\end{itemize}
			Technologies : Python, Ruby, HTML, XML, XSL, GNU/Linux, Cluster de calcul, calcul scientifique, C, Shell
		}
%		\cventry{Juillet 2012\\à Août 2012}{Développeur web}{Australian Fuel Cells}{Gold-Coast}{Australie}{
%			\begin{itemize}
%				\item Développement du site internet de l'entreprise.
%			\end{itemize}
%			Technologies : HTML, JavaScript
%		}
%		\cventry{Septembre 2008\\à Août 2009}{Analyste développeur}{Logica}{Lyon}{France}{
%			\begin{itemize}
%				\item Analyse technique des erreurs survenues lors des traitements batch pour le compte du Groupama Système d'Information.
%			\end{itemize}
%			Technologies : PacBase, MVS, JCL, DB2, COBOL
%		}
%		\cventry{Janvier 2008\\à Juin 2008}{Développeur web}{AU-PC Informatique}{Montréal}{Canada}{
%			\begin{itemize}
%				\item Développement d'applications web, agenda, gestion des annomalies ...
%			\end{itemize}
%			Technologies : PHP, HTML, JavaScript, CakePHP, apache
%		}
        \cventry{Septembre 2006\\à Septembre 2011}{Développeur en embarquée}{Clubelek}{Lyon}{France}{
			Membre du club de robotique de l'INSA de Lyon pour la participation à la coupe de France et d'Europe.
			\begin{itemize}
				\item Développement de capteurs et actionneurs communicant sur un bus CAN.
				\item Développement d'une IHM embarquée.
				\item Contribution au développement de l'I.A..
			\end{itemize}
			Technologies : Assembleur, C, $\mu$C/OS II, CAN, SPI, USART, PIC18F, M32, STM32
		}
	\section{Formations}
		\cventry{2009-2012}{Ingénieur en Informatique et Réseaux de Communication}{CPE}{Lyon}{}{Spécialité développement embarqué}
		\cventry{2006-2008}{D.U.T Informatique}{Université Claude Bernard Lyon 1 et CEGEP de Rosemont à Montreal}{}{}{}
		\cventry{2005-2006}{Bac Scientifique avec mention}{Lycée Edouart Branly}{Lyon 5}{}{Option Sciences de l'Ingénieur - Spécialité Maths}
	\section{Compétences}%
		\subsection{Informatique}%
			\cvcomputer{Langages}{C, C++, Python, Ruby, MongoDB, SQL, XML, XSL, HTML, JS, Shell, \LaTeX, Java}{Systèmes d'exploitations}{GNU/Linux, Mac OS X, $\mu$C/OS II, Windows}%
			\cvcomputer{Micro-contrôleurs}{PIC18F, M32, STM32}{Réseaux}{TCP/IP, CAN, SPI, USART, SIP, SCCP, ICMP, RTP}%
			\cvcomputer{Modélisation}{Merise, UML}{Gestion de projet}{Méthode en V, AGILE}
		\subsection{Langues}
			\cvlanguage{Français}{Langue maternelle}{}
			\cvlanguage{Anglais}{Lu, parlé, écrit}{Niveau CECR B2}
			\cvlanguage{Espagnol}{Notions}{}
	\section{Centres d'intérêt}
		\cvline{Sports}{Cirque, natation, roller, ski, plongée sous-marine et surf.}
		\cvline{Arts}{Cirque et théâtre.}
\end{document}
